\hypertarget{index_intro_sec}{}\section{Introduction}\label{index_intro_sec}
This is the introduction to iteration 3 of the interpreter project. So far we have created the scanner and parser for the interpreter. The scanner will read from a file and create a linked list of tokens that all contain Enumerated Tokentypes and using these Enumerated Tokentypes the parser is then able to generate an Abstract Syntax Tree (A\+ST). The linked list of tokens is passed to the parser and using the Tokentypes is able to parse them into an A\+ST and with each Node in the A\+ST is able to unparse which will generate c++ code equivalent to the F\+C\+AL language we are interpreting from\hypertarget{index_Scanner}{}\subsection{Scanner}\label{index_Scanner}
The scanner reads in characters from another file and and using regex expressions the scanner is able to categorize which characters are which Enumerated Tokentype. At the same time the scanner is also scanning for white space which it gets rid of using the regex for white space and bypasses the white space by moving the pointer reading the input file. After each character is properly categorized it is placed as a Token type in a linked last.\hypertarget{index_Parser}{}\subsection{Parser}\label{index_Parser}
The Parser reads in the Token linked list from the scanner and goes through each Token in the linked list and generates a subclass according to the Token\+Type of each Token in the linked list. The first class generated is always the Root class which is the root of the A\+ST that will be generate by the Parser. After this Root class has been generated other Stmt, Stmts, Expr, and Decl subclasses will be generated according to the Token\+Types of the rest of the Tokens in the Token linked list that was passed by the Scanner. 